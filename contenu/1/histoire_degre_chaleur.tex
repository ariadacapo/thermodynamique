\atstartofhistorysection
\onlyamphibook{\section[Un peu d’histoire : mesurer le degré de chaleur]{Un peu d’histoire :\\ mesurer le degré de chaleur}}
\onlyframabook{\section{Un peu d’histoire : mesurer le degré de chaleur}}
\label{ch_histoire_degre_chaleur_depondt}

\begin{center}\textit{Par Philippe Depondt\\ \begin{small}Université Pierre et Marie Curie, Paris\\ Contribution sous licence \ccbysa\end{small}}\end{center}

	Pour Aristote, au \textsc{iv}\ieme siècle avant J.C. en Grèce, le feu était l'un des quatre constituants de la matière avec l'eau, l'air et la terre. L'idée de mesurer quelque chose, le feu ou autre, c'est-à-dire de donner une valeur numérique à une grandeur, lui était parfaitement étrangère car sa physique était essentiellement non-mathématique~\cite{koyre1966} : ses théories étaient basées sur des observations \emph{qualitatives}. Or la synthèse des idées d'Aristote avec le christianisme avait été faite au \textsc{xii}\ieme siècle par Thomas d'Aquin et, depuis, ces idées étaient très largement dominantes dans le monde savant en Europe jusqu'au début du \textsc{xvii}\ieme siècle. Galilée en particulier devra, en effet, se déterminer en grande partie \emph{contre} ces idées.
	
	Les descriptions du monde restaient donc pour l'essentiel qualitatives. Les astronomes faisaient toutefois exception : depuis les Babyloniens, ils observaient le ciel, mesuraient aussi précisément que possible la position des étoiles et des planètes à des fins de prédictions astrologiques ou pour des raisons d'ordre religieux. Copernic, dans les premières années du \textsc{xvi}\ieme siècle s'appuie, pour établir son modèle héliocentrique du Monde, sur les mesures remontant à l'Antiquité, puis d'astronomes arabes du Moyen-Âge. Le premier «~laboratoire~» moderne, avec des instruments spécialisés dont la précision et la qualité est dûment vérifiée, un personnel qualifié, des protocoles d'observation rigoureux, des compte-rendus de mesure systématiques, etc., est l'observatoire que fit construire l'astronome danois Tycho Brahé sur l'île de Hveen entre le Danemark et la Suède à la fin du \textsc{xvi}\ieme siècle. Ce sont ces mesures remarquablement complètes et d'une extrême précision (moins d'une minute d'angle) qui ont permis la découverte par Johannes Kepler de ses trois lois qui constituèrent un des fondements de la dynamique de Newton.
	
	Dans le cas spécifique de la chaleur, Francis Bacon, le théoricien de la méthode inductive au début du \textsc{xvii}\ieme siècle, prend justement la chaleur comme exemple pour illustrer son propos. Il propose ainsi, dans le \textit{Novum Organum}, de recenser toutes les observations dans lesquelles la chaleur apparaît, toutes celles où elle n'apparaît pas et enfin celles où elle apparaît «~par degré~».\\
	Toutefois, les observations «~à la Bacon~» restent purement qualitatives ; mais, à peu près au même moment, on assiste à une explosion des tentatives de mesure réellement quantitatives du degré de chaleur.
	
	Il semble que le premier thermomètre ait vu le jour vers 1605 entre les mains d'un hollandais nommé Cornelius Drebbel~\cite{locqueneux1996} : basé sur des idées remontant à Héron d'Alexandrie (\textsc{i}\up{er}\xspace siècle après \textsc{j.c.}), il était constitué d'une sphère creuse en verre prolongée d'un tube orienté vers le bas et plongé dans un liquide coloré. Si la sphère était chauffée, le liquide était chassé vers le bas par dilatation de l'air, et au contraire, si elle était refroidie le liquide montait dans le tube : c'était donc un thermomètre à air. Ce thermomètre servit un peu plus tard à suivre la fièvre chez des malades, mais il avait l'inconvénient d'être aussi sensible aux variations de pression atmosphérique qu'à la température. Vers le milieu du siècle, les thermomètres à liquide s'avérèrent beaucoup plus fiables et aussi plus faciles d'emploi. La sphère de verre se trouvait désormais placée en bas du dispositif et était remplie d'un liquide coloré qui montait dans un tube gradué ; ce tube était d'abord ouvert, puis il apparut qu'en le fermant, on évitait l'évaporation du liquide. Ces perfectionnements avaient été fortement soutenus par la grand-duc Ferdinand II de Medicis et ces dispositifs furent ainsi appelés «~thermomètres de Florence~».
	Restait le problème des graduations. Le nombre de graduations était assez variable, 50, 60, 100..., les artisans se bornant à tenter de reproduire ce qu'ils avaient eux-mêmes déjà fait et dans le meilleur des cas des thermomètres construits par la même personne donnaient à peu près le même résultat. Faute d'échelle universellement acceptée,il était donc impossible de pouvoir réaliser des mesures en divers lieux avec des appareils différents pour les comparer.
	Dans les premières années du \textsc{xviii}\ieme siècle, Guillaume Amontons construit un thermomètre à air basé sur la mesure d'une différence de pression et non de volume (le volume du gaz change peu, la section du tube étant petite). Amontons ayant observé que si l'on continuait à chauffer de l'eau bouillante, son degré de chaleur n'augmentait pas, il utilise cette référence comme point fixe. Il fallait évidemment corriger les mesures par une mesure simultanée de la pression atmosphérique. Ce système permet à Amontons de faire une découverte majeure : si la pression augmente quand le degré de chaleur augmente, à l'inverse, elle diminue quand le degré de chaleur diminue ; au minimum, cette pression devient nulle, ainsi que le degré de chaleur. Le degré de chaleur correspondant est ainsi obtenu, en unités modernes, pour \SI{-239,5}{\degreeCelsius} ! Une première mesure du zéro absolu.
	
	Ces thermomètres restent toutefois d'un emploi délicat qui limite considérablement leur diffusion.	René-Antoine Ferchault de Réaumur, vers le milieu du \textsc{xviii}\ieme siècle, met au point un thermomètre à mélange eau-alcool dans lequel le degré d'alcool est précisément fixé afin d'assurer la reproductibilité de l'instrument. Il le gradue en choisissant deux références : la glace fondante et l'eau bouillante et il divise cet intervalle en 80 degrés. Cette échelle est appelée «~échelle de Réaumur~». 
	
	En 1724, à Dantzig, Daniel Gabriel Fahrenheit décrit un thermomètre qui utilise la dilatation du mercure et introduit une échelle pour laquelle la glace fondante est à \SI{32}{\degree} et la température du sang à \SI{96}{\degree} ; un mélange de glace, d'eau et de sel d'ammoniac lui donne le zéro de son échelle.
	
	En 1741,le suédois Anders Celsius reprend l'échelle de Réaumur mais la divise en 100 intervalles au lieu de 80 : cette échelle est assez largement diffusée en France et en 1794, au moment de l'adoption du système métrique par la Convention, c'est l'échelle de Celsius qui est adoptée comme échelle officielle.

	Le passage de la sensation subjective de chaud et de froid à la mesure objective de la température avec des instruments fiables et une échelle universelle, entraîna un grand nombre de constatations qui n'allaient jusqu'alors pas de soi : la température d'une cave n'est pas plus élevée en hiver qu'en été, le fer n'est pas plus froid que le bois, etc., et, somme toute, tout cela est assez récent !

\atendofhistorysection
