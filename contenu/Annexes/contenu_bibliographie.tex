\textbf{Pour étudier la thermodynamique de l’ingénieur :}

\begin{itemize}
	\item Pour compléter ce livre et aller plus loin, en français : Çengel, Boles \& Lacroix 2007~\mbox{\cite{cengeletal2008en,cengeletal2008fr}}, une couverture accessible, exhaustive et solide des sujets abordés ici ;\index{Cengel@Çengel, Boles \& Lacroix}
	\item Pour compléter ce livre et aller plus loin, en anglais : Eastop \& McConkey 1993~\cite{eastopetal1993}, incontournable référence (lui-même basé sur le très robuste Rogers \& Mayhew 1992~\cite{rogersetal1992}) ;\index{Eastop@Eastop \& McConkey}\index{Rogers@Rogers \& Mayhew}
	\item Pour approfondir : Watzky 2007~\cite{watzky2007}, une approche plus analytique et rigoureuse que celle qui est menée ici.\index{Watzky, Alexandre}
\end{itemize}

\textbf{Pour découvrir la thermodynamique en physique :}

\begin{itemize}
	\item Depondt 2001~\cite{depondt2001}, une exploration ludique de notre discipline préférée (Philippe Depondt a écrit les sections \S\ref{ch_histoire_degre_chaleur_depondt}, \S\ref{ch_histoire_quantite_chaleur_depondt}, \S\ref{ch_histoire_lavoisier_laplace_depondt}, et \S\ref{ch_histoire_rumford_depondt} de ce manuel) ;\index{Depondt, Philippe}
	\item Feynman 1963~\cite{feynman1963,feynman1963fr} aborde la thermodynamique à plusieurs reprises dans son livre introductif de physique (devenu une référence), en particulier avec une belle exploration de la notion d’irréversibilité.\index{Feynman, Richard}
\end{itemize}

\textbf{Pour voyager dans le temps :}

\begin{itemize}
	\item Gay-Lussac 1807~\cite{gaylussac1807} et Joule 1845~\cite{joule1845} pour un acompte du travail minutieux et méticuleux qui a permis de poser les bases de ce que l’on appelle aujourd’hui \textit{premier principe} ;\index{Gay-Lussac!Louis Joseph}\index{Joule!James Prescott}
	\item Carnot 1824~\cite{carnot1824}, le livret d’une modernité époustouflante du premier ingénieur thermodynamicien de l’histoire ;\index{Carnot!Sadi}
	\item Clausius 1854~\cite{clausius1854} pour voir naître le terme «~entropie~» et voir l’aplomb de son papa.\index{Clausius!Rudolf}
\end{itemize}

{\center \normalsize \Large Références\par}

\printbibliography[heading=none]
