\newgeometry{left=5.25cm,right=5.25cm,top=1.5cm,bottom=1.5cm,includeheadfoot}
\thispagestyle{empty}

\vspace*{\stretch{1}}
{{\color{gray}\scalebox{3}{«}} Nous avons étudié ce sujet avec tout l’intérêt, nous pourrions dire, avec tout l’entraînement qu’il excitait en nous. Quel tableau admirable en effet que ce triomphe de l’intelligence humaine ! Quel imposant spectacle que celui d’une locomotive se mouvant sans effort apparent et tirant derrière elle un train de~40 ou~50 voitures chargées, pesant chacune dix~milliers de~livres ! Que sont désormais les plus lourds fardeaux avec des machines qui peuvent mouvoir des poids si énormes ; que sont les distances avec des moteurs qui franchissent journellement un intervalle de 12~lieues en  1½~heure ? Le sol disparaît en quelque sorte sous vos yeux ; les arbres, les maisons, les montagnes, sont entraînés derrière vous avec la rapidité d’un trait, et lorsque vous croisez un autre train avec une vitesse relative de 15 à 20 lieues à l’heure, vous l’apercevez en un moment poindre, grandir et vous toucher ; et à peine l’avez-vous vu passer avec effroi, que déjà il est emporté loin de vous, devenu un point, et disparu de nouveau dans le lointain. \raisebox{-1.6ex}{{\color{gray}\scalebox{3}{»}}}}

{\small\begin{flushright}
François-Marie Guyonneau de Pambour\\
\textit{Traité théorique et pratique des machines locomotives}~\begin{scriptsize}\cite{pambour1835}\end{scriptsize}, 1835
\end{flushright}}
\vspace{1cm}
\vspace*{\stretch{1}}
\restoregeometry
