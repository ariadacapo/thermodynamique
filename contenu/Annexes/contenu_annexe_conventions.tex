
\TabPositions{2cm}
\begin{description}
	\item[$\equiv$] 	\tab Par définition. Le symbole $\equiv$ pose la définition du terme à sa gauche (qui ne dépend donc pas d’équations antérieures).
	\item[$\dot~$]		\tab (point au dessus d’un symbole) Débit dans le temps : $\dot~ \equiv \frac{\diff}{\diff t}$. Par exemple $\dot Q$ est le débit de chaleur (en \si{watts}) représentant une quantité $Q$ (en \si{joules}) par \si{seconde}.
	\item[$\Delta$]	\tab dénote une différence nette entre deux valeurs : $(\Delta X)_\fromatob = X_\B - X_\A$. Elle peut être négative.
	\item[italiques] 	Propriétés physiques (masse $m$, température $T$).
	\item[indices]		En caractères droits : points dans le temps ou dans l’espace (température $T_\A$ au point A). «~cste.~» abrège «~constante~», «~in~» dénote «~entrant~» et «~out~» «~sortant~».
	\item[minuscules]	Valeurs spécifiques (énergie ou puissance). Voir \S\ref{ch_valeurs_spécifiques}.
	\item[opérateurs]	Différentielle $\diff$, exponentielle $\exp x \equiv e^x $, logarithme naturel $\ln x \equiv \log_e x$
	\item[unités]		Les unités sont en caractères droits et en gris (\SI{1}{\kilogram}). Dans les phrases les unités sont en toutes lettres et conjuguées (cent \si{watts}). Le \si{litre} est noté \si{\liter} pour le rendre plus lisible ($\SI{1}{\liter} \equiv \SI{e-3}{\metre\cubed}$). Les unités des équations sont celles du système international d’unités (\textsc{si}) sauf indication.
	\item[nombres]		Le séparateur de décimale est la virgule, le multiplicateur d’exposant est un point, les chiffres des entiers sont groupés par trois ($\SI{1,234e3} ~=~ \num{1234}$). Les arrondis sont effectués aussi tard que possible et jamais en série ; les zéros de début et de fin ne sont jamais indiqués.
\end{description}
