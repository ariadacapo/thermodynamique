
\TabPositions{2cm}

Les conventions graphiques et de signe sont décrites dans les sections \S\ref{ch_convention_signe_sf} p.\pageref{ch_convention_signe_sf}, \S\ref{ch_conventions_signe_so} p.\pageref{ch_convention_signe_so}, et \S\ref{ch_conventions_graphiques} p.\pageref{ch_conventions_graphiques}.

\begin{description}
	\item[$\equiv$] 	\tab Par définition. Le symbole $\equiv$ pose la définition du terme à sa gauche (qui ne dépend donc pas d’équations antérieures).
	\item[$\dot~$]		\tab (point au dessus d’un symbole) Débit dans le temps : $\dot~ \equiv \frac{\diff}{\diff t}$. Par exemple $\dot Q$ est le débit de chaleur (en \si{watts}) représentant une quantité $Q$ (en \si{joules}) par \si{seconde}.
	\item[$\Delta$]	\tab dénote une différence nette entre deux valeurs : $(\Delta X)_\fromatob = X_\B - X_\A$. Elle peut être négative.
	\item[italiques] 	Propriétés physiques (masse $m$, température $T$).
	\item[indices]		En caractères droits : points dans le temps ou dans l’espace (température~$T_\A$ au point A). Les indices «~cst.~» et «~cste~» dénotent une propriété qui reste constante, l’indice «~rév.~» indique que le calcul est effectué le long d’une évolution réversible, «~in~» dénote «~entrant~» et «~out~» «~sortant~».\\
							En caractères italiques : $T_H$, $T_B$, et les indices $TH$ et $TB$ dénotent une température haute ou basse, comme détaillé en \S\ref{ch_limites_machines_thermiques} p.\pageref{ch_limites_machines_thermiques}. Les indices~$L$ et~$V$ indiquent les points de saturation d’un mélange liquide-vapeur, comme détaillé en \S\ref{ch_points_saturation} p.\pageref{ch_points_saturation}.
	\item[minuscules]	Valeurs spécifiques (énergie ou puissance). Voir \S\ref{ch_valeurs_spécifiques} p.\pageref{ch_valeurs_spécifiques}.
	\item[opérateurs]	Différentiel $\diff$, exponentielle $\exp x \equiv e^x $, logarithme naturel $\ln x \equiv \log_e x$ ;
	\item[unités]		Les unités sont en caractères droits et en gris (\SI{1}{\kilogram}). Dans les phrases les unités sont en toutes lettres et conjuguées (cent \si{watts}). Le \si{litre} est noté \si{\liter} pour le rendre plus lisible ($\SI{1}{\liter} \equiv \SI{e-3}{\metre\cubed}$). Les unités des équations sont celles du système international d’unités (\textsc{si}) sauf indication contraire.
	\item[nombres]		Le séparateur de décimale est la virgule, le multiplicateur d’exposant est un point, les chiffres des entiers sont groupés par trois ($\SI{1,234e3} ~=~ \num{1234}$). Les arrondis sont effectués aussi tard que possible et jamais en série ; les zéros de début et de fin ne sont jamais indiqués.
\end{description}
