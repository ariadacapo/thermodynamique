		
		\index{pression!indiquée}\index{pression!réelle}\index{pression!atmosphérique}\index{atmosphère, pression de l’}\index{manomètre!jaugé}
		
		La pression indiquée sur un baromètre n’est pas toujours la pression réelle du fluide.
		
		En effet, on mesure souvent la pression à l’aide de manomètres qui sont calibrés sur la pression atmosphérique. Par exemple, lorsque l’on regonfle un pneu en station-service, le manomètre indique \SI{0}{\bar} à pression ambiante --\ toutes les pressions qu’il indiquera seront décalées de la valeur de la pression atmosphérique. Nous nommons cette valeur indiquée la \textbf{\vocab[pression!jaugée]{pression jaugée}}.

		La pression jaugée se définit comme :
		\begin{equation*}
		p_{\text{j}} \equiv  p_{\text{réelle}} - p_{\text{atm.}}
		\end{equation*}
		
		\onlyframabook{\begin{footnotesize}}%handmade, pour éviter une cassure disgracieuse
		\begin{equationterms}
			\item où \tab $p_{\text{j}}$ 			\tab\tab\tab\tab est la pression jaugée, indiquée au cadran (\si{\pascal}),
			\item 	\tab $p_{\text{réelle}}$	\tab est la pression réelle au sein du réservoir où est faite la mesure (\si{\pascal}),
			\item et \tab $p_{\text{atm.}}$ 		\tab\tab est la pression atmosphérique ambiante (\si{\pascal}).
		\end{equationterms}
		\onlyframabook{\end{footnotesize}}

		Un manomètre jaugé indique donc \SI{0}{\bar} quelle que soit la pression ambiante, s’il est laissé à l’air libre. La pression indiquée lors d’une mesure dépendra de la pression atmosphérique ambiante ; elle peut être positive (pneu) ou parfois négative (conduit d’eau ou de pétrole).
		
		La pression jaugée est intéressante parce qu’elle indique \emph{la différence} de pression entre chacun des côtés des parois du réservoir (pneu, canalisation) ; elle révèle donc les contraintes qu’elles subissent. La pression réelle de l’air à l’intérieur du pneu n’a pas d’intérêt en tant que telle pour l’automobiliste.

		En revanche, c’est la pression réelle dont nous avons besoin pour prédire l’état des fluides. Nous utilisons donc \textbf{toujours la pression réelle} dans nos calculs.
