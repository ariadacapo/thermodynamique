La thermodynamique est l’étude de la conversion de l’énergie entre deux formes, chaleur et travail. Pourtant, ses débuts remontent bien avant que ces trois concepts ne soient établis : pendant longtemps il ne s’agissait que de se pencher sur \emph{la nature de la chaleur}. Autrement dit, que veut dire «~chaud~» exactement ? Peut-on le mesurer ?

Les premières réflexions sur la nature de la matière et celle du feu datent de la Grèce antique et donnent déjà naissance à la théorie atomique. Mais il ne s’agit alors que de constructions philosophiques, plus fondées sur une vision spirituelle organisée du monde que sur de réels travaux d’observation.

Il faudra attendre le \textsc{xvii}\ieme siècle pour que débutent de sérieux travaux de recherche sur ce sujet. C’est la température, dont on se fait plus facilement une idée que de la chaleur, qui est d’abord le centre d’intérêt. La conception du thermomètre soulève en effet de nombreux problèmes d’ingénierie et de physique : comment lier cette idée de «~température~» à un phénomène observable directement, de façon prévisible et reproductible ?

Pendant ces années et jusqu’en 1850, la thermodynamique reste à l’échelle macroscopique –~il n’est pas encore question d’atome ou de molécule. Elle suscite beaucoup d’intérêt parce qu’elle aborde directement les phénomènes de frottement et de transfert de chaleur, qui ne se produisent jamais que dans un seul sens, et auxquels une vision mécanique newtonienne de l’univers ne peut fournir d’explication.

Le grand essor des machines thermiques, au début du \textsc{xix}\ieme siècle, prend la science de court. Les premiers moteurs pompent l’eau hors des mines, mais la thermodynamique –\ qui ne porte alors même pas son nom\ – ne sait pas expliquer comment. Il faudra une trentaine d’années avant que la théorie ne rattrape la pratique et que l’on établisse une vision cohérente de la thermodynamique permettant, par exemple, de prévoir le rendement d’un moteur.

En 1865, le physicien allemand Rudolf Clausius clôture près d’un siècle de tâtonnements en explicitant les grandes bases de ce que l’on commence à appeler «~thermodynamique~» : c’est ce que nous connaissons aujourd’hui sous le nom des deux principes. Il généralise, ce faisant, ses observations sur un ballon de gaz à l’univers tout entier.
De leur côté, l’écossais James Clerk Maxwell et l’autrichien Ludwig Boltzmann réconcilieront la thermodynamique avec la physique des particules en travaillant au niveau microscopique. Au fur et à mesure du \textsc{xx}\ieme siècle, le concept d’incertitude se fait accepter et la thermodynamique devient affaire de probabilités et de quantification du désordre ; elle sert même à poser les bases de la théorie de l’information.

Entre temps, la révolution industrielle a eu lieu. Délaissant la pompe à eau, le moteur thermique est passé à la propulsion des locomotives, puis des navires, automobiles, génératrices de courant et aéronefs. Notre mode de vie, dans lequel la force physiologique humaine n’a plus la moin\-dre importance, montre à quel point nous som\-mes devenus dépendants de la puissance et de la précision que ce moteur permet. En somme, il est la raison pour laquelle notre environnement diffère tant de celui de nos ancêtres, et de celui que connaîtront nos descendants. La thermodynamique permet de comprendre le fonctionnement déroutant de cet engin à la fois banal et effroyable.

Au cours de cette série de dix chapitres sur la \textit{thermodynamique de l’ingénieur}, nous passerons du comportement élémentaire des fluides à la théorie des moteurs –\ l’objectif étant de fournir à l’étudiant/e une bonne compréhension du fonctionnement des machines à chaleur et une base solide pour pouvoir aborder la conception moteur et la mécanique des fluides.

