Lors du \courssix, nous avons étudié la nature de différents cycles permettant de transformer chaleur et travail. Nous nous proposons maintenant d’étudier, d’expliquer et de quantifier leurs limites. Ce \courssept se propose de répondre à deux questions :
\begin{itemize}
	\item Pourquoi tous les moteurs thermiques ont-ils toujours un rendement inférieur à~\SI{100}{\percent} ?
	\item Comment maximiser le rendement d’une machine thermique ?
\end{itemize}
