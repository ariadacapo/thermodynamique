\section{Pourquoi utiliser un système fermé ?}
\index{système!fermé, utilité de}\index{système!ouvert, utilité de}

		À partir de maintenant, nous voulons décrire et quantifier les transferts d’énergie dans les fluides. Nous pouvons adopter deux points de vue différents pour observer le fluide :
		\begin{itemize}
			\item Soit nous «~découpons~» un petit morceau de masse, que nous suivons de près lorsqu’il évolue, puis nous quantifions l’énergie qui lui est transférée : c’est ce que nous appelons un \vocab[système!fermé vs. ouvert]{système fermé} ;
			\item Soit nous choisissons un morceau de volume fixe, qui est \emph{traversé} en permanence par un débit de masse, puis nous quantifions les transferts d’énergie vers le volume : c’est ce que nous appelons un \vocab[système!fermé vs. ouvert]{système ouvert}.
		\end{itemize}
		
		Bien sûr, ces deux méthodes sont équivalentes : elles vont produire les mêmes résultats. Le choix de l’une ou l’autre rendra simplement l’analyse et la quantification des transferts plus aisée.

		\begin{figure}
			\begin{center}
			\includegraphics[width=8cm]{images/diesel_engine_cutaway.jpg}
			\end{center}
			\supercaption{Une découpe dans un moteur de camion laisse apparaître trois pistons dans leurs cylindres. Un système fermé est un bon outil pour étudier l’air emprisonné dans un cylindre. Le moteur photographié est un diesel \textsc{man} en V8.}%
				{\wcfile{Cutaway of a MAN V8 Diesel engine.jpg}{Photo} \ccbysa \olivier}
				\label{fig_diesel_engine_cutaway}
		\end{figure}
		
		L’utilisation d’un système fermé est judicieuse pour analyser les machines à mouvement alternatif (moteurs automobiles, pompes et compresseurs et généralement toutes les machines à pistons/cylindres). Ces machines divisent le fluide en petites quantités qui sont emprisonnées dans une enclave, dans laquelle elles sont chauffées, refroidies, comprimées ou détendues (\cref{fig_diesel_engine_cutaway}). Il est alors facile d’identifier une quantité de masse donnée et de quantifier les transferts qu’elle subit.
		
		À l’inverse, pour étudier ce qui se passe dans une tuyère de turboréacteur par exemple, nous aurions des difficultés pour identifier un groupe donné de particules et quantifier le changement de ses propriétés. Il sera alors judicieux d’utiliser un système ouvert, comme nous l’étudierons au \courstrois.

		\index{moteur!principe de fonctionnement}		
		Concrètement, dans ce chapitre, nous voulons quantifier le travail qu’on peut générer avec un fluide dans un cylindre. Un moteur de voiture fournit du travail parce que l’air dans les cylindres fournit plus de travail en se détendant au retour qu’il n’en a reçu à en se faisant comprimer à l’aller (\cref{fig_moteur_sf}). Comment peut-on générer cela ? Pour répondre à cette question, il nous faut une méthode robuste pour quantifier les transferts d’énergie.

		\onlyamphibook{\begin{figure}}
		\onlyframabook{\begin{figure}[tbh]}%handmade pour éviter de trop gros blancs
			\begin{center}
			\includegraphics[width=\textwidth]{images/fonctionnement_base_moteur.png}
			\end{center}
			\supercaption{Principe de fonctionnement d’un moteur. Lorsque l’on fournit de la chaleur à un fluide dans un réservoir fermé, celui-ci augmente les forces qu’il exerce sur les parois du réservoir. En laissant le réservoir se déformer, on fait effectuer un travail au fluide.}{schéma \cczero \oc}
			\label{fig_moteur_sf}
		\end{figure}


\section{Conventions de comptabilité}
\label{ch_conventions_compta_sf}

	\subsection{Le système fermé}

			Nous appelons \vocab[système!fermé, définition]{système fermé} un sujet d’étude arbitraire dont les frontières sont imperméables à la masse : un ensemble donné de particules, de masse fixe. Toutes les propriétés de cet ensemble (pression, température, volume, etc.) peuvent être amenées à changer, mais il s’agit toujours des mêmes molécules, non mélangées à d’autres. Par exemple, un gaz prisonnier dans un cylindre et comprimé par un piston (\cref{fig_piston_m_cste}) est parfaitement décrit avec un système fermé.

		\onlyframabook{\begin{figure}[tbh]}%handmade pour éviter de trop gros blancs
		\onlyamphibook{\begin{figure}[tbh!]}%handmade là il faut y aller carrément!
			\begin{center}
			\onlyframabook{\includegraphics[width=6cm]{images/piston_cylindre.png}}
			\onlyamphibook{\includegraphics[width=5cm]{images/piston_cylindre.png}}
			\end{center}
			\supercaption{Un système fermé typique : une quantité de masse fixe dans un réservoir fermé.
Une paroi mobile permet de la comprimer ; nous lui ferons également recevoir et perdre de la chaleur.}{schéma \cczero \oc}
			\label{fig_piston_m_cste}
		\end{figure}
		
	\onlyamphibook{\vspace{-1em}}
	\subsection{Conventions de signe}
	\label{ch_convention_signe_sf}
	\index{système!férmé, conventions de signe}

		Pour quantifier les transferts nous utiliserons la convention de signe suivante, illustrée en \cref{fig_convention_signe_sf} :

		\begin{itemize}
			\item Lorsqu’ils sont positifs, les transferts $Q$ et $W$ traduisent une \emph{réception} par le système.
			\item À l’inverse, lorsqu’ils sont négatifs, les transferts $Q$ et $W$ indiquent une \emph{perte} du système. Le travail $W$ est alors fourni et la chaleur $Q$ émise.
		\end{itemize}

		\onlyamphibook{\begin{figure}}
		\onlyframabook{\begin{figure}[tbh]}%handmade pour éviter de trop gros blancs
			\begin{center}
				\includegraphics[width=9cm]{images/convention_systeme_ferme.png}
			\end{center}
			\supercaption{Conventions de signe pour un système fermé. Les flux entrants sont positifs, les flux sortants sont négatifs ; ils sont tous représentés avec des flèches rentrantes. La quantité de masse est fixe.}{schéma \cczero \oc}
			\label{fig_convention_signe_sf}
		\end{figure}

		Ainsi, dans les équations, nous pouvons systématiquement additionner les termes sans avoir à connaître le sens des transformations. Les transferts sont comptabilisés comme sur un compte bancaire : les dépenses sont négatives et les recettes positives.


\section{Le premier principe dans un système fermé}
\label{ch_premier_principe_sf}
\index{système!férmé, premier principe}

	Le premier principe stipule que l’énergie est indestructible (\S\ref{ch_premier_principe}). Si on fournit \SI{100}{\joule} de travail à un système fermé et qu’il perd \SI{80}{\joule} sous forme de chaleur, c’est donc que «~son~» énergie a augmenté de~\SI{20}{\joule}. Nous nommons cette augmentation la \vocab[énergie!interne, variation de]{variation d’énergie interne}, $\Delta U$.
	
	Sous forme d’équation, le premier principe dans un système fermé se traduit par l’équation :
	\begin{equation}
		Q_{1 \to 2} + W_{1 \to 2} = \Delta U
		\label{eq_premier_principe_sf_maj}
	\end{equation}
	\begin{equationterms}
		\item pour un système fermé immobile ;
		\item où \tab $\Delta U = U_2 - U_1$ est la variation d’énergie interne (\si{\joule}),
		\item 	\tab $W_{1 \to 2}$ 	\tab est le travail reçu par le système (\si{\joule}),
		\item et \tab $Q_{1 \to 2}$ 	\tab est la chaleur reçue par le système (\si{\joule}).
	\end{equationterms}

	\onlyframabook{\thermoquotetopbegin{O}}\onlyamphibook{\thermoquotebegin{O}}\index{Clausius!Rudolf}
		Appelons ainsi $Q$ la quantité totale de chaleur qui doit être impartie à un corps pendant son passage, d’une manière donnée, depuis une condition à une autre, (toute chaleur prélevée au corps étant comptée comme une quantité négative), alors nous la divisons en trois parties, parmi lesquelles la première est employée à augmenter la chaleur existant véritablement dans le corps, la seconde à produire le travail intérieur et la troisième à produire le travail extérieur. Il va de la seconde partie ce que nous avons déjà dit de la première : qu’elle est indépendante du chemin suivi dans le passage du corps d’un état à un autre, et nous pouvons en conséquence représenter ces deux parties ensemble par une fonction $U$, qui même si nous ne pouvons mieux la définir, nous savons à l’avance au moins être entièrement déterminée par les états initial et final du corps.
	\thermoquoteend{Rudolf Clausius, 1854}{\textit{Über eine veränderte Form des zweiten Hauptsatzes der mechanischen Wärmetheorie~\cite{clausius1854}}\vspace{0em}} %handmade vspace
	Malheureusement l’énergie interne $U$ est parfois très difficile à mesurer. Nous verrons dans les chapitres~\quatre et~\cinq que les corps emmagasinent cette énergie interne de différentes façons, et qu’elle est intimement liée à la température. L’énergie interne $U$, par définition, est toujours positive, mais sa variation $\Delta U$ ne l’est pas nécessairement.

	L’\cref{eq_premier_principe_sf_maj} peut être exprimée avec des grandeurs spécifiques :
	\begin{IEEEeqnarray}{rCl}
		m \ ( q_{1 \to 2} + w_{1 \to 2} )		& = & m \ \Delta u \nonumber \\
		q_{1 \to 2} + w_{1 \to 2} 				& = & \Delta u
	\label{eq_premier_principe_sf_min}
	\end{IEEEeqnarray}
		\onlyframabook{\begin{footnotesize}}%handmade
	\begin{equationterms}
		\item pour un système fermé immobile ;
		\item où \tab $\Delta u = u_2 - u_1$ est la variation d’énergie interne spécifique\onlyamphibook{ (\si{\joule\per\kilogram})},
		\item 	\tab $w_{1 \to 2}$ 	\onlyamphibook{\tab} est le travail spécifique reçu par le système (\si{\joule\per\kilogram}),
		\item et \onlyamphibook{\tab }$q_{1 \to 2}$ 	\onlyamphibook{\tab\tab }est la chaleur spécifique reçue par le système (\si{\joule\per\kilogram}).
	\end{equationterms}%handmade, lotsof
	\onlyframabook{\end{footnotesize}}
	
	Nous pouvons encore ré-écrire cette \cref{eq_premier_principe_sf_min} pour l’exprimer sous sa \vocab[principes de la thermodynamique!premier, forme différentielle]{forme différentielle} :
	\begin{IEEEeqnarray}{rCl}
		\diff q + \diff w	& = & \delta u
	\label{eq_premier_principe_sf_diff}
	\end{IEEEeqnarray}
	\begin{equationterms}
		\item pour un système fermé immobile ;
		\item où \tab $\delta u$ 	\tab\tab est la variation infinitésimale d’énergie interne spécifique (\si{\joule\per\kilogram}),
		\item 	\tab $\diff w$ 	\tab est le transfert infinitésimal de travail spécifique (\si{\joule\per\kilogram}),
		\item et \tab $\diff q$ 	\tab est le transfert infinitésimal de chaleur spécifique (\si{\joule\per\kilogram}).
	\end{equationterms}

	Dans cette \cref{eq_premier_principe_sf_diff}, les opérateurs $\diff$ et $\delta$ ont le même sens mathématique (celui de quantités infinitésimales) mais des significations phyiques différentes : $\diff w$ représente un \emph{transfert} infinitésimal qui s’intégrera en $w_{1 \to 2}$, tandis que $\delta u$ représente une \emph{variation} infinitésimale qui s’intégrera en $\Delta u = u_2 - u_1$.
	
	\index{cycle!thermodynamique}
	Lorsqu’un fluide est ramené à son état initial (même pression, même volume, même température), alors il contient exactement la même quantité d’énergie interne qu’auparavant. La totalité de l’énergie qu’il a reçue (sous forme de chaleur ou de travail) a donc nécessairement été rendue à l’extérieur sous une forme ou une autre. Nous exprimons cette affirmation ainsi :
	\begin{equation}
	Q_{\text{cycle}} + W_{\text{cycle}} = 0
	\label{eq_premier_principe_cycle}
	\end{equation}
	\begin{equationterms}
		\item pour un cycle thermodynamique complet,
		\item où \tab $W_\text{cycle}$ \tab est le travail reçu par le système (\si{\joule}),
		\item et \tab $Q_\text{cycle}$ \tab est la chaleur reçue par le système (\si{\joule}).
	\end{equationterms}

	\index{principes de la thermodynamique!premier, appliqué à un cycle}
	Cette \cref{eq_premier_principe_cycle} est la raison pour laquelle on énonce souvent le premier principe —\ sans pourtant apporter grand’chose à notre simple affirmation du \coursunshort\ — de la façon suivante :

	«~Lorsqu’un système a parcouru un cycle thermodynamique complet, la somme algébrique de la chaleur fournie et du travail effectué est nulle.~»



\section[Quantifier le travail avec un système fermé]{Quantifier le travail avec un système\onlyamphibook{\\} fermé}
\index{travail!calcul en système fermé}

	Le calcul du travail avec les fluides est délicat. Nous allons procéder en trois étapes~de complexité croissante :

	\begin{itemize}
		\item En remplaçant le fluide par un ressort ;
		\item En comprimant le fluide de façon infiniment lente ;
		\item En comprimant le fluide de façon rapide.
	\end{itemize}



	\subsection{Le travail en fonction du volume, avec un ressort}
	\label{ch_travail_pdv}
	\index{ressorts!modélisation des gaz}

		\begin{figure}
			\begin{center}
				\includegraphics[width=10cm]{images/travail_cylindre_1.png}
			\end{center}
			\supercaption{Dans un premier temps, nous modélisons le fluide à l’intérieur du système avec un ressort métallique.}{schéma \ccbysa \olivier}
			\label{fig_piston_ressort}
		\end{figure}

		Commençons par imaginer que le fluide au sein d’un système fermé se comporte comme un ressort métallique (\cref{fig_piston_ressort}). C’est une modélisation intéressante pour débuter notre étude. Nous avions vu en \S\ref{ch_travail_fdl} que le travail fourni ou reçu par un ressort s’exprimait selon :
		\begin{equation*}
			W_\fromatob = - \int_\A^\B {F \diff l} \tag{\ref{eq_travail_fdl}}
		\end{equation*}

		Aujourd’hui, comme nous utilisons un fluide, nous voulons exprimer le travail en fonction des propriétés \vocab{pression} et \vocab{volume} plutôt que force et longueur.

		\clearfloats
		\begin{description}

			\item[La pression]{se définit comme une force divisée par une aire :
			\begin{equation}
				p \equiv \frac{F}{S}
				\label{def_pression}
			\end{equation}
			\begin{equationterms}
				\item où \tab $p$ \tab est la pression (\si{\pascal}),
				\item 	\tab $F$ \tab est la force (\si{\newton}),
				\item et \tab $S$ \tab est l’aire de la surface sur laquelle la force s’applique (\si{\metre\squared}).
			\end{equationterms}

			L’unité \textsc{si} de la pression est le~\si{Pascal},
			\begin{equation}
				\SI{1}{\pascal} \equiv \SI{1}{\newton\per\metre\squared}
			\end{equation}
			mais il est usuel d’utiliser le~\si{bar} pour unité :
			\begin{equation}
			\SI{1}{bar} \equiv \SI{1e5}{\pascal}
			\end{equation}

			\index{pression!jaugée}\index{pression!atmosphérique}
			Notons que la pression atmosphérique à faible altitude est de l’ordre du bar ($p_{\text{atm.std.}} \equiv \SI{1}{atm} \equiv \SI{1,01325}{bar}$). Attention, les manomètres indiquent souvent une pression jaugée, qui n’est pas la pression réelle. Cette différence est décrite dans l’annexe \ref{ch_annexe_pression}.
			}% end item

			\item[Le volume]{est également exprimable facilement. Si le système est déformé par un piston d’aire $S$, de sorte que sa longueur varie de $\diff l$, nous avons :
			\begin{equation}
			\diff V = S \diff l
			\label{eq_volume_surface_longueur}
			\end{equation}
			\begin{equationterms}
				\item où \tab $\diff V$ 	\tab est la variation infinitésimale du volume (\si{\metre\cubed}),
				\item 	\tab $S$ 			\tab\tab l’aire du piston déplacé (\si{\metre\squared}),
				\item et \tab $\diff l$ 	\tab\tab la variation infinitésimale de longueur du système correspondant au déplacement du piston (\si{\metre}).
			\end{equationterms}

			Dans le système d’unités \textsc{si} le volume se mesure en~\si{\metre\cubed} mais l’unité de mesure la plus courante est le~\si{litre} ($\SI{1}{\liter} \equiv \SI{e-3}{\metre\cubed})$.
			}% end item
		
		\end{description}

		\index{ressorts!compression de}
		Exprimons maintenant le travail d’un système fermé en fonction du volume et de la pression. En insérant les équations~\ref{def_pression} et~\ref{eq_volume_surface_longueur} dans l’\cref{eq_travail_fdl} nous obtenons :
		\begin{IEEEeqnarray}{rCl}
			W_\fromatob 	& = & - \int_\A^\B {F \diff l} = - \int_\A^\B {\frac{F}{S} S \diff l} 	\nonumber \\
			W_\fromatob 	& = & - \int_\A^\B {p \diff V} \label{eq_travail_pdV}
		\end{IEEEeqnarray}
		\begin{equationterms}
			\item pour un système fermé modélisé par un ressort,
			\item où \tab $W_\fromatob$ 	est le travail reçu par le système (\si{\joule}),
			\item 	\tab $p$ 				\tab\tab est la pression (homogène) intérieure (\si{\pascal}),
			\item et \tab $\diff V$ 		\tab la variation du volume (\si{\metre\cubed}).
		\end{equationterms}


		Pour pouvoir quantifier l’énergie stockée ou fournie par le système, il nous suffira donc de connaître la relation entre $p$ et $V$. Dans le cas présent, cette fonction $p_{(V)}$ est directement liée à la caractéristique $F_{(l)}$ du ressort. La dureté du ressort et sa géométrie (à spires régulières ou progressives) détermineront au final la quantité de travail stockée et fournie par le système.
		
		Un outil formidable pour comprendre et analyser les transferts de chaleur est le \vocab[diagramme!pression-volume]{diagramme pression-volume}.\index{pression-volume, diagramme} Dans le cas où l’on modélise le fluide par un ressort, le travail peut être visualisé par l’aire sous la courbe d’une évolution (\cref{fig_p-v_ressort}).		

		\onlyamphibook{\begin{figure}}
		\onlyframabook{\begin{figure}[h!]}
			\begin{center}
				\includegraphics[width=8cm]{images/pv_ressort.png}
			\end{center}
			\supercaption{Diagramme pression-volume d’un système fermé modélisé par un ressort. Dans le cas représenté, le volume augmente (le piston s’éloigne). La grandeur $\diff V$ sera constamment positive, et le travail sera négatif : le système perd de l’énergie au profit du piston.\\
			Cette figure représente le même phénomène que celui de la \cref{fig_force-déplacement-aire} p.\pageref{fig_force-déplacement-aire}, avec des grandeurs physiques différentes.}{schéma \cczero \oc}
			\label{fig_p-v_ressort}
		\end{figure}

\onlyframabook{\clearpage}
		\index{ressorts!compression de}
		\begin{anexample}
			
			Un système fermé est constitué d’une boîte vide dans laquelle on a placé un ressort. La pression exercée par le ressort sur les parois de la boîte est constante à $p = \SI{e5}{\pascal}$ quelque soit son volume. On comprime la boîte depuis un volume $V_\A = \SI{2}{\liter}$ jusqu’à $V_\B = \SI{1}{\liter}$. Quel est le transfert de travail ?
				\begin{answer}
						Sur un diagramme pression-volume et de façon qualitative (c’est-à-dire sans représenter les valeurs numériques), l’évolution peut être représentée ainsi :
							\begin{center}
								\includegraphics[width=3cm]{images/exe_pv_isobare.png}
							\end{center}
					Nous partons de l’équation \ref{eq_travail_pdV} : $W_{\A\to\B} = - \int_\A^\B p \diff V = - p_\text{cste.} \int_\A^\B \diff V = p_\text{cste.} \left[V\right]_{V_\A}^{V_\B } = - \num{e5} (\num{1e-3} - \num{2e-3}) = \SI{+100}{\joule}.$
					\begin{remark}Le signe est positif : la boîte («~le système~») reçoit du travail. Nous explicitons toujours le signe lorsque nous quantifions les transferts.\end{remark}
				\end{answer}
		\end{anexample}

		\index{ressorts!compression de}
		\begin{anexample}
			Un système fermé a une pression interne liée à son volume par la relation $p = \num{7e5} - \num{2e8} \ V$ (en unités \textsc{si}). On comprime la boîte depuis un volume $V_\A = \SI{2}{\liter}$ jusqu’à $V_\B = \SI{1}{\liter}$. Combien a-t-il reçu ou perdu d’énergie sous forme de travail ?
				\begin{answer}
						Sur un diagramme pression-volume et de façon qualitative, l’évolution peut être représentée ainsi :
							\begin{center}
								\includegraphics[width=3cm]{images/exe_pv_prop.png}
							\end{center}
					Encore une fois nous partons de l’\cref{eq_travail_pdV} : $W_{\A\to\B} = - \int_\A^\B p \diff V = - \int_\A^\B (\num{7e5} - \num{2e8} V) \diff V = - \left[\num{7e5}V - \frac{1}{2}\num{2e8} V^2\right]_{V_\A}^{V_\B } = - (\num{700} - 100 - \num{1400} + 400) = \SI{+400}{\joule}$ (positif : travail reçu par le système).
				\end{answer}
		\end{anexample}
		

\onlyframabook{\clearpage}%handmade
	\subsection{Travail d’un fluide en évolution lente}
	\index{travail!en système fermé, en évolution lente}\index{système!fermé, travail en évolution lente}

		\onlyframabook{\thermoquotebegin{O}%handmade dans le framabook, la citation plus haut qu’elle ne devrait l’être, sinon tout est cassé
\index{Clapeyron! Benoît Paul Émile}%
			Cela posé, prenons un gaz quelconque à la température $T$ \jecourte; représentons son volume $v_o$ par l’abscisse \textsc{ab}, et sa pression par l’ordonnée \textsc{cb}.\jecourte Le gaz, pendant sa dilatation, aura développé une quantité d’action mécanique qui aura pour valeur l’intégrale du produit de la pression, par la différentielle du volume, et qui sera représentée géométriquement par la surface comprise entre l’axe des abscisses, les deux coordonnées \textsc{cb}, \textsc{de}, et de la portion d’hyperbole~\textsc{ce}.
		\thermoquoteend{Benoît Paul Émile Clapeyron, 1834\\ {\tiny(le premier diagramme $p-v$…)}}{\textit{Mémoire sur la puissance motrice de~la~chaleur}~\cite{clapeyron1834}\vspace{3em}}} %handmade vspace
		Lorsque l’on comprime un fluide, les molécules qui le composent sont plus rapprochées les unes des autres (\cref{fig_molécules_compression_lente}) et les collisions entre elles et contre les parois deviennent plus fréquentes. À l’échelle macroscopique, cette augmentation se traduit par une augmentation de la pression.
		
		\begin{figure}
			\begin{center}
			\includegraphics[width=9cm]{images/particules_compression_lente.png}
			\end{center}
			\supercaption{Une représentation simpliste d’un fluide que l’on \mbox{comprime} infiniment lentement sans le chauffer. Le fluide voit sa température et sa pression augmenter.}{schéma \cczero \oc}
			\label{fig_molécules_compression_lente}
		\end{figure}

		Nous constatons expérimentalement que lorsque le mouvement est infiniment lent, un fluide comprimé se comporte exactement comme un ressort (\cref{fig_piston_fluide_lent}). La précision «~lorsque le mouvement est infiniment lent~» est d’importance capitale, comme nous le verrons plus~bas.

		\begin{figure}
			\begin{center}
			\includegraphics[width=8cm]{images/travail_cylindre_2.png}
			\end{center}
			\supercaption{Lorsque le mouvement du piston est infiniment lent, le fluide se comporte comme un ressort que l’on comprime.}{schéma \ccbysa \olivier}
			\label{fig_piston_fluide_lent}
		\end{figure}

		
		Si cette condition est respectée, nous pouvons exprimer le travail reçu ou perdu par le système de la même façon qu’avec le ressort de la section précédente :		
		\begin{IEEEeqnarray}{rCl}
			W_\fromatob 	& = & - \int_\A^\B {p \diff V}	\nonumber \\
			w_\fromatob 	& = & - \int_\A^\B p \diff v
			\label{eq_travail_pdv}
		\end{IEEEeqnarray}
		\begin{equationterms}
			\item pour un système fermé lorsque les variations de volume sont infiniment lentes ;
			\item où \tab $w_\fromatob$ 	\tab est le travail spécifique reçu par le système (\si{\joule\per\kilogram}),
			\item 	\tab $p$ 				\tab\tab est la pression (homogène) intérieure (\si{\pascal}),
			\item et \onlyamphibook{\tab} $\diff v $ 		\onlyamphibook{\tab} la variation du volume spécifique (\si{\metre\cubed\per\kilogram}). %handmade floating quote messes with the tabs in Framabook, so they are commented out here.
		\end{equationterms}
		
		\onlyamphibook{\thermoquotebegin{O}%handmade la citation là où elle devrait bien être.
\index{Clapeyron, Benoît Paul Émile}
			Cela posé, prenons un gaz quelconque à la température $T$ \jecourte; représentons son volume $v_o$ par l’abscisse \textsc{ab}, et sa pression par l’ordonnée \textsc{cb}.\jecourte Le gaz, pendant sa dilatation, aura développé une quantité d’action mécanique qui aura pour valeur l’intégrale du produit de la pression, par la différentielle du volume, et qui sera représentée géométriquement par la surface comprise entre l’axe des abscisses, les deux coordonnées \textsc{cb}, \textsc{de}, et de la portion d’hyperbole~\textsc{ce}.
		\thermoquoteend{Benoît Paul Émile Clapeyron, 1834\\ {\tiny(le premier diagramme $p-v$…)}}{\textit{Mémoire sur la puissance motrice de~la~chaleur}~\cite{clapeyron1834}\vspace{6em}}} %handmade vspace
		Sur un graphique représentant la pression en fonction du volume spécifique, ce travail $w_\fromatob$ est représenté par la surface sous la courbe de A à B, exactement comme la \cref{fig_p-v_ressort}. La forme de la courbe, c’est-à-dire la relation entre $p$ et $v$ au fur et à mesure de l’évolution, déterminera la quantité $w_\fromatob$ .

		\begin{figure}
			\begin{center}
			\includegraphics[width=\pvdiagramwidth]{images/pv_gaz_simple.png}
			\end{center}
			\supercaption{Propriétés d’un gaz lorsqu’on le comprime, représentées sur un diagramme pression-volume. La relation est similaire à celle que l’on obtiendrait avec un ressort à spires progressives.}{schéma \cczero \oc}
			\label{fig_p-v_pvx}
		\end{figure}

		\onlyframabook{\thermoquotetopbegin{O}}\onlyamphibook{\thermoquotebegin{O}}%handmade
\index{Pambour, François-Marie Guyonnau de}
			S’il était vrai que la vapeur se dépensât par le cylindre à une pression égale à celle de la chaudière, ou qui fût à celle-ci dans un rapport fixe indiqué par un coefficient quelconque, puisqu’il faut toujours à une même locomotive le même nombre de tours de roue, ou le même nombre de coups de piston pour parcourir la même distance, il s’en suivrait que tant que ces machines travaillent à la même pression, elles devraient consommer dans tous les cas la même quantité d’eau pour la même distance.
		\thermoquoteend{François-Marie Guyonneau de~Pambour, 1839}{\textit{Théorie de la machine à vapeur} \cite{pambour1839}\onlyamphibook{\vspace{4em}}\onlyframabook{\vspace{-2em}}} %handmade vspace
\index{ressorts!modélisation des gaz}
		Comment les fluides se comportent-t-ils lorsqu’on les comprime --\ autrement dit, par quel type de «~ressort~» peut-on les modéliser ? On constate expérimentalement que, lorsqu’on les comprime, la plupart des gaz voient leur pression et leur volume liés par une relation de type $p\ v^{x} = \text{cste.}$ avec $x$ une constante (\cref{fig_p-v_pvx})%
			\footnote{Toutefois, nous verrons au \courscinqshort que sur une plage de propriétés donnée, les liquides/vapeurs se comportent de façon très différente, même si la tendance globale reste la même.}%
		.

		Lorsqu’on apporte de la chaleur au fluide pendant qu’on le comprime, on «~durcit~» son comportement, et la pression augmente plus rapidement (\cref{fig_p-v_ajout_retrait_chaleur}). À~l’inverse, lorsqu’on lui prélève de la chaleur pendant la compression, la pression augmente moins rapidement. Ces transferts de chaleur font donc varier la quantité de travail à fournir pour comprimer le fluide entre deux volumes donnés.
		
		Le cas où l’on apporte pas de chaleur est nommé \vocab[évolutions!adiabatiques]{\mbox{adiabatique}}\index{adiabatiques, évolutions} : $Q = 0$. Attention : adiabatique ne veut pas dire «~à température constante~». Lorsque l’on comprime un fluide sans apport de chaleur, sa température augmente. Dans un moteur diesel, par exemple, l’air dans les cylindres peut atteindre \SI{900}{\degreeCelsius} avant la combustion -- ce qui est désirable, comme nous le verrons au \courssept.

		\begin{figure}
			\begin{center}
			\includegraphics[width=\didacticpvdiagramwidth]{images/pv_transfert_chaleur.png}
			\end{center}
			\supercaption{Comportement d’un fluide lorsqu’on le comprime infiniment lentement.
Plus on lui apporte de chaleur pendant la compression, plus la pression augmente fortement. La courbe adiabatique représente le cas où aucun apport ni perte de chaleur n’a lieu ($Q = 0$).}{schéma \cczero \oc}
			\label{fig_p-v_ajout_retrait_chaleur}
		\end{figure}


		Dans les trois évolutions de la \cref{fig_p-v_ajout_retrait_chaleur}, la relation de type $p v^{x} = \text{cste.}$ reste une modélisation appropriée. Plus on apporte de chaleur pendant la compression, plus la pression augmente rapidement --\ l’exposant $x$ est alors plus important.

		Inversement, si l’on prélève de la chaleur pendant la compression, la pression augmente moins rapidement et on obtient une courbe plus proche de l’horizontale (avec un exposant $x$ plus faible). En prélevant suffisamment de chaleur, on peut même maintenir la pression constante, comme nous le verrons aux chapitres~\quatre et~\cinq. L’exposant $x$ est alors nul et on a $p = p_\text{cste.}$.
		
			\begin{anexample}
				Un gaz dans un cylindre est comprimé lentement par un piston. On observe que sa pression est liée à son volume par la relation $p v^{\num{1,2}} = k$ (en unités \textsc{si}, et où $k$ est une constante). Au début de la compression, ses propriétés sont $p_\A = \SI{1}{\bar}$ et $v_\A = \SI{1}{\metre\cubed\per\kilogram}$. On le comprime jusqu’à ce que son volume ait atteint $v_\B = \SI{0,167}{\metre\cubed\per\kilogram}$. \\
				Quelle quantité de travail spécifique le gaz a-t-il fourni ou reçu ?				
					\begin{answer}
						Sur un diagramme pression-volume et de façon qualitative, l’évolution peut être représentée ainsi :
							\begin{center}
								\includegraphics[width=3cm]{images/exe_pv_exp1.png}
							\end{center}
						Il nous faut d’abord calculer la valeur de $k$ pour connaître quantitativement la relation entre $p$ et $v$. Nous l’obtenons avec les conditions initiales : $k = p_\A v_\A^{\num{1,2}} = \num{e5} \times 1^{\num{1,2}} = \SI{e5}{\usi}$.
							\begin{remark} La grandeur de $k$ est déroutante : elle est mesurée en~\si{\pascal\metre\tothe{3,6}\per\kilogram\tothe{1,2}}. Cela n’a pas d’importance pour nous et il nous suffit (après avoir bien converti les unités d’entrée en \textsc{si}!) d’indiquer «~unités \textsc{si}~», ou~\si{\usi}.\end{remark}
						Maintenant, nous pouvons décrire $p$ en fonction de $v$ : $p = \num{e5} \times v^{\num{-1,2}}$. Il n’y a plus qu’à intégrer en partant de l’\cref{eq_travail_pdv} : $w_\fromatob = - \int_\A^\B p \diff v = - \int_\A^\B k v^{\num{-1,2}} \diff v = - k \left[\frac{1}{-1,2 + 1} v^{-1,2 + 1} \right]_{v_\A}^{v_\B } = \frac{\num{e5}}{0,2}\left[v^{-0,2}\right]_1^{\num{0,167}} = \SI{+2,152e5}{\joule\per\kilogram} = \SI{+215,2}{\kilo\joule\per\kilogram}$.
							\begin{remark}Le signe de $w_\fromatob$ est positif : le gaz reçoit du travail.\end{remark}
							\begin{remark}Le résultat peut paraître grand, mais il faut se rappeler que c’est un travail spécifique (\S\ref{ch_valeurs_spécifiques}) qu’il faudra multiplier par la masse du gaz pour obtenir une quantité en joules. Aux conditions de départ (\SI{1}{\kilogram\per\metre\cubed}) un volume d’air de~\SI{1}{\liter} pèse à peine plus d’un~\si{gramme}. \end{remark}
					\end{answer}
			\end{anexample}

			\begin{anexample}
				Une masse de~\SI{0,3}{gramme} de gaz pressurisée dans un cylindre est détendue lentement en laissant un piston se déplacer. On sait que sa pression et son volume sont reliés par une relation de type $p v^{k_1} = k_2$ (où $k_1$ et $k_2$ sont deux constantes).\\
				Au début de la détente, la pression est à~\SI{12}{\bar} et le volume est de~\SI{0,25}{\liter}. Une fois détendu, le gaz arrive à pression ambiante de~\SI{1}{\bar} avec un volume de~\SI{1,76}{\liter}.\\
				Quel travail le gaz a-t-il dégagé pendant la détente ?				
					\begin{answer}
						Sur un diagramme pression-volume et de façon qualitative, l’évolution peut être représentée ainsi :
							\begin{center}
								\includegraphics[width=3cm]{images/exe_pv_exp2.png}
							\end{center}
						Il nous faut d’abord connaître entièrement la loi reliant $p$ à $v$ ; ensuite nous procéderons à l’intégration $-\int p \diff v$ pendant l’évolution pour calculer le travail.\\
						Commençons par calculer les volumes spécifiques au départ et à l’arrivée : $v_\A = \frac{V_\A}{m} = \frac{\num{0,25e-3}}{\num{3e-4}} = \SI{0,833}{\metre\cubed\per\kilogram}$. De même, $v_\B = \frac{V_\B }{m} = \SI{5,867}{\metre\cubed\per\kilogram}$.
						Maintenant, nous pouvons calculer $k_1$ :		
							\begin{IEEEeqnarray*}{rCl}
								p_\A v_\A^{k_1} 	&=& p_\B v_\B ^{k_1}	\\
								\left(\frac{v_\A}{v_\B }\right)^{k_1} &=& \frac{p_\B }{p_\A}\\
								k_1 \ln\left(\frac{v_\A}{v_\B }\right) &=& \ln \left(\frac{p_\B }{p_\A}\right)\\
								k_1 &=& \frac{\ln\left(\frac{p_\B }{p_\A}\right)}{\ln\left(\frac{v_\A}{v_\B }\right)} = \frac{\ln\left(\frac{1}{12}\right)}{\ln\left(\frac{0,833}{5,867}\right)} = \num{1,2733}
							\end{IEEEeqnarray*}
						Et avec $k_1$, nous pouvons calculer $k_2 = p_\A v_\A^{k_1} = \num{12e5} \times \num{0,833}^{\num{1,2733}} = \SI{9,514e5}{\usi}$.
							\begin{remark}$k_1$ est un exposant et n’a pas d’unités. Les unités de $k_2$ ne nous intéressent~pas.\end{remark}
							\begin{remark}Même si elle peut paraître laborieuse, cette démarche «~nous avons un modèle général pour la tendance, quels sont les paramètres pour ce cas particulier ?~» est très courante en physique, et extrêmement utile pour l’ingénieur/e.\end{remark}
						Nous savons maintenant décrire quantitativement les propriétés pendant l’évolution : $p v^{\num{1,2733}} = \num{5,914e5}$.	Il n’y a plus qu’à effectuer notre intégration habituelle : $w_\fromatob = - \int_\A^\B p \diff v = - k_2 \int_\A^\B v^{-k_1} \diff v = \frac{\num{-9,514e5}}{\num{-0,2733}} \left[v^{-0,2733}\right]_{\num{0,833}}^{\num{0,587}} = \SI{-3,333e6}{\joule\per\kilogram} = \SI{-3333}{\kilo\joule\per\kilogram}$. Nous multiplions par la masse de gaz pour obtenir le travail : $W_{\A\to\B} = m \ w_\fromatob = \SI{-1}{\kilo\joule}$.
							\begin{remark}Ce calcul peut être effectué de façon plus rapide, sans calculer les valeurs de $v_\A$, $v_\B $ et $k_2$. Toutefois, pour être certain/e de parvenir au résultat, il est plus sûr et plus facile de quantifier $p$ et $v$ (en \textsc{si}) à tous les stades de l’évolution avant de débuter une intégration.\end{remark}
					\end{answer}
			\end{anexample}
			
			\begin{anexample}
				Un gaz enfermé dans un réservoir hermétique est chauffé lentement. Son volume reste bloqué à~\SI{12}{\liter}, et sa pression évolue de~\SI{1}{\bar} jusqu’à \SI{40}{\bar}. Quel est le travail développé ?
					\begin{answer}
						Sur un diagramme pression-volume et de façon qualitative, l’évolution peut être représentée ainsi :
							\begin{center}
								\includegraphics[width=3cm]{images/exe_pv_isochore.png}
							\end{center}
						Le travail est nul, bien sûr. Le volume ne changeant pas, $\diff V$ est nul pendant toute l’évolution. Nous pouvons chauffer ou refroidir à loisir, mais tant qu’aucune paroi n’est déplacée, il n’y aura pas de transfert de travail.
					\end{answer}
			\end{anexample}
			

	\subsection{Travail d’un fluide en évolution rapide}
	\label{ch_évolutions_irr_sf}
	\index{travail!en système fermé, en évolution rapide}\index{système!fermé, travail en évolution rapide}

		\thermoquotebegin{O}\index{Pambour, François-Marie Guyonnau de}
			Nous avons dit qu’à l’origine du mouvement l’équilibre de pression s’établit entre la chaudière et le cylindre, mais à mesure que la vitesse du piston s’accroît, celui-ci fuit en quelque sorte devant la vapeur sans lui donner le temps d’établir cet équilibre, et la pression dans le cylindre baisse \mbox{nécessairement}.
		\thermoquoteend{François-Marie Guyonneau de~Pambour, 1835}{\textit{Traité théorique et pratique des machines locomotives}~\cite{pambour1835}\vspace{1em}} %handmade vspace
		Les choses se compliquent lorsque nous comprimons et détendons notre fluide de façon rapide (\cref{fig_molécules_rapide}). Il se produit alors un phénomène complexe et d’importance critique en thermodynamique : \textbf{la pression sur la paroi est différente de la «~pression moyenne~» à l’intérieur du fluide}.

		\begin{figure}
			\begin{center}
			\includegraphics[width=9cm]{images/particules_compression_rapide.png}
			\end{center}
			\supercaption{Compression et détente irréversibles. Lorsqu’on comprime un fluide de façon brutale (schéma de gauche), la pression sur la paroi du piston est augmentée. Lors d’une détente brutale (schéma de droite) cette pression est diminuée.}{schéma \ccbysa \olivier}
			\label{fig_molécules_rapide}
		\end{figure}

		Pour décrire ce qui se passe à l’intérieur du fluide, nous pouvons prendre l’exemple de l’eau d’une baignoire que l’on pousse avec les mains --\ comme l’objet représenté en \cref{fig_baignoire} qui est déplacé dans de l’eau liquide. Lorsque l’objet est éloigné et rapproché brutalement, la pression sur ses parois n’est pas la même que lorsqu’il est déplacé lentement.

		\begin{figure}[htb!]%handmade: Là il faut poser les figures avant de continuer, sinon ça devient trop dur à suivre.
			\begin{center}
				\includegraphics[width=\textwidth]{images/mouvement_rapide_niveau_eau.png}
			\end{center}
			\supercaption{Un objet solide déplacé dans un réservoir d’eau, avec un mouvement lent (en haut) et avec un mouvement rapide (en bas). Lors du mouvement rapide, les forces de pression aidant le mouvement sont plus faibles, et les forces s’opposant au mouvement sont plus grandes. Dans le cas limite où le mouvement est infiniment lent, ces forces sont égales.}{schéma \cczero \oc}
			\label{fig_baignoire}
		\end{figure}
		
		\onlyframabook{\clearfloats\clearpage\pagebreak\par}%handmade
		Dans chacun des cas, la quantité de travail consommé à la compression est plus grande et la quantité de travail fourni à la détente est plus petite.
		
		Nous nommons ce phénomène l’\vocab{irréversibilité}. Elle nous sera d’un grand embarras dans notre étude quantitative de la thermodynamique et elle rendra encore plus ardues nos conversions de travail et chaleur.

		Que se passe-t-il donc dans le cylindre empli de fluide, lorsqu’on ne le comprime pas de façon infiniment lente ? Lors d’une compression brutale, la pression sur la paroi du piston est plus grande que la pression moyenne à l’intérieur du cylindre (\cref{fig_piston_fluide_rapide}). On dépense \emph{plus d’énergie que nécessaire} pour effectuer le déplacement.

		\begin{figure}
			\begin{center}
				\includegraphics[width=8cm]{images/travail_cylindre_3.png}
			\end{center}
			\supercaption{Fluide comprimé de façon brutale. La pression locale à la surface du piston est supérieure à ce qu’elle aurait été avec un mouvement lent.}{schéma \ccbysa \olivier}
			\label{fig_piston_fluide_rapide}
		\end{figure}

\index{ressorts!modélisation des gaz}
		On pourrait ainsi dire que lorsqu’on le comprime et détend brutalement, un fluide se comporte comme un ressort «~fragile~», à l’intérieur duquel quelque chose se modifie : il n’est pas capable de rendre toute l’énergie mécanique qu’il a emmagasinée.

\index{chaleur!transformation de travail en}\index{travail!transformation en chaleur}
		Si le travail reçu n’est pas égal au travail restitué, alors où est passé l’excédent d’énergie ? Ce surcroît d’énergie, fourni sous forme de travail par le piston, est \emph{transformé en chaleur à l’intérieur du fluide} pendant les mouvements.

		\onlyamphibook{\clearfloats}%handmade encore une fois, sinon c’est trop b****lique
		L’évolution tracée sur un diagramme pression-volume (\cref{fig_p-v_compression_irr}) est bien plus complexe que dans le cas d’une évolution infiniment lente. La pression moyenne à l’intérieur du fluide augmente plus rapidement qu’elle ne le ferait en mouvement~lent.

		\begin{figure}
			\begin{center}
				\includegraphics[width=\didacticpvdiagramwidth]{images/pv_compression_irreversible.png}
			\end{center}
			\supercaption{Compression irréversible adiabatique sur un diagramme pression-volume.\\
				Nous représentons l’évolution du gaz en pointillés : il ne s’agit pas d’une série d’états continue car la pression du fluide n’est pas homogène pendant le trajet.\\
				Le chemin qu’aurait suivi le fluide si la compression avait été infiniment lente est représenté avec un trait continu.\\
				Pendant la compression, le «~surcoût~» de travail fourni par le piston est transformé en chaleur (bien que le gaz soit parfaitement isolé).}{schéma \cczero \oc}
			\label{fig_p-v_compression_irr}
		\end{figure}

		Pendant la détente, le phénomène inverse se produit (\cref{fig_p-v_détente_irr}) : une zone de plus faible pression se forme au devant de la paroi du piston, et le travail fourni par le fluide au piston est plus faible qu’il ne l’aurait été dans le cas réversible.

		\begin{figure}
			\begin{center}
			\includegraphics[width=9cm]{images/pv_detente_irreversible.png}
			\end{center}
			\supercaption{Détente irréversible adiabatique sur un diagramme pression-volume.\\ Le travail reçu par le piston est plus faible qu’il ne l’aurait été avec un mouvement lent. Le trajet suivi par le fluide est représenté en pointillés (la pression n’étant pas homogène pendant le mouvement).}{schéma \cczero \oc}
			\label{fig_p-v_détente_irr}
		\end{figure}

		D’un point de vue quantitatif, plus les mouvements sur le fluide seront brutaux, et plus l’évolution du fluide ressemblera à une évolution avec apport de chaleur («~durcissement~» du fluide et augmentation de l’exposant $x$ pendant les compressions, diminution de l’exposant $x$ pendant les détentes).

		Par contre, le travail fourni ou reçu par le fluide ne peut plus être simplement calculé par intégrale puisque la pression à l’intérieur du cylindre n’est pas du tout homogène. C’est la pression à la surface du piston qui permettrait de calculer ce travail. Malheureusement, aucune relation mathématique simple ne permet de décrire cette relation entre pression et volume. Il faut effectuer une mesure expérimentale à chaque fois.
		
			
			\begin{anexample}
				On enferme un gaz dans un cylindre hermétique et on effectue des allers-retours entre deux volumes donnés avec le piston, sans transférer de chaleur. Au départ, les allers-retours sont très lents. Ensuite, on effectue les allers-retours de façon très rapide.
				
				Quelle sera l’allure des évolutions sur un diagramme pression-volume ?
					\begin{answer}
						Lors des évolutions lentes, la pression passe toujours par les mêmes valeurs pendant les allers-retours :
							\begin{center}
								\includegraphics[width=3cm]{images/exe_pv_rev.png}
							\end{center}
						En revanche, pendant les évolutions rapides, à chaque trajet la pression finale est plus grande que ce qu’elle aurait été pendant un trajet lent :
							\begin{center}
								\includegraphics[width=3cm]{images/exe_pv_irr.png}
							\end{center}
						Ainsi, le volume s’élève progressivement sur le diagramme pression-volume : l’excédent de travail investi pendant les compressions, et le défaut de travail récupéré pendant la détente, se traduisent par une augmentation de l’énergie interne du gaz, dont la température augmente continuellement.
					\end{answer}
			\end{anexample}

\onlyframabook{\clearpage}%handmade
	\subsection{La réversibilité}
	\label{ch_reversibilite}\index{réversibilité}\index{irréversibilité!vs. réversibilité}
	
		Prenons quelques instants pour réfléchir sur ce que nous venons de décrire. À chaque fois que nous comprimons un fluide «~trop vite~», il se passe quelque chose qui nous empêche de récupérer notre travail.
		
		Du point de vue de l’ingénieur/e, une évolution lente est un cas limite : celui où les dissipations sont minimisées. Par exemple, le travail qu’il faut investir pour comprimer un gaz jusqu’à~\SI{10}{\bar} est minimal lorsque la compression est réversible. De même, une turbine dans laquelle la détente est réversible extraira le maximum de travail d’un fluide comprimé. Et au contraire, dans un amortisseur automobile, on rend les évolutions très irréversibles pour qu’il fournisse lors du chemin retour un travail plus faible qu’à l’aller.
		
		\thermoquotebegin{O}
		D’où vient l’irréversibilité ? elle ne vient pas des lois de Newton. Si nous partons de l’idée que le comportement de toutes choses doit être en définitive compris en termes des lois de la physique et s’il apparaît également que toutes les équations ont cette propriété fantastique d’avoir une autre solution [valide] lorsque nous remplaçons $t$ par $-t$, alors chaque phénomène est réversible. Comment se fait-il alors que dans la nature, à une grande échelle, les choses ne soient pas réversibles ?
		\thermoquoteend{Richard Feynman, 1963}{\textit{The Feynman Lectures on \mbox{Physics}} \mbox{\cite{feynman1963, feynman1963fr}}\onlyamphibook{\vspace{-2em}}\index{Feynman, Richard}} %handmade vspace
		Du point de vue de la physique, le phénomène d’irréversibilité est fascinant. En effet, nous partons de collisions de molécules, un phénomène tout à fait réversible, pour fabriquer une transformation irréversible : une évolution qui ne va que dans un sens ! Pour ramener le gaz dans l’état où il était avant de le comprimer brutalement, nous sommes obligés de lui prendre de la chaleur. Il est surprenant que sans aller à l’encontre des lois de Newton, nous ayons créé une situation où \emph{on ne peut pas revenir en arrière en «~faisant l’inverse~»}. Existe-t-il d’autres transformations irréversibles ? Peut-on quantifier l’irréversibilité ? Nous tenterons de répondre à ces questions dans les chapitres \coursseptplural et \courshuitplural.

		En attendant, nous admettrons qu’il faut respecter trois conditions pour qu’une évolution soit réversible :

		\begin{enumerate}
			\item L’évolution doit se faire sans friction. Il ne doit pas y avoir de frottement dans les éléments mécaniques (par exemple, entre piston et cylindre).
			\item La pression dans le fluide doit être homogène. Le mouvement des parois doit donc être infiniment lent, et le fluide doit évoluer sans turbulence.
			\item La différence de température entre le fluide et son environnement doit être infiniment petite. Si de la chaleur est fournie ou rejetée, elle doit donc être transférée de façon infiniment lente.
		\end{enumerate}

		Ces trois conditions excluent évidemment toute évolution réelle --\ et en particulier, toute application pratique dans un moteur ! Toutefois, nous les utiliserons pour poser une limite théorique idéale à toutes les évolutions réelles que nous étudierons.


\onlyframabook{\clearpage}
\section[Quantifier la chaleur avec un système fermé]{Quantifier la chaleur avec un système\onlyamphibook{\\} fermé}
\index{chaleur!calcul en système fermé}\index{système!fermé, calcul de chaleur}

		Au risque de frustrer l’étudiant/e, il nous faut tout de suite avouer que \emph{nous ne savons pas quantifier directement les transferts de chaleur}. Nous allons toujours procéder par déduction : en quantifiant la variation d’énergie, et en y soustrayant les transferts sous forme de travail, on obtient la quantité de chaleur qui a été transférée. Mathématiquement, dans un système fermé, nous ne faisons que réutiliser l’\cref{eq_premier_principe_sf_maj} pour obtenir :
	\begin{IEEEeqnarray}{rCl}
		Q_{1 \to 2} 	& = & 	\Delta U \ - \  W_{1 \to 2} \\
		q_{1 \to 2} 	& = & 	\Delta u \ - \  w_{1 \to 2}
	\end{IEEEeqnarray}
	\begin{equationterms}
		\item pour un système fermé.
	\end{equationterms}
		
		Toute la difficulté pour quantifier un transfert de chaleur est maintenant de prédire et quantifier le changement de l’énergie interne, $\Delta U$. Pour les gaz, $U$ est quasiment proportionnelle à la température ; pour les liquides et vapeurs, la relation est plus complexe. Nous apprendrons à quantifier l’énergie dans les fluides aux chapitres \coursquatreplural et \courscinqplural.
