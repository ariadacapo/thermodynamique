\atstartofhistorysection
\section[Un peu d’histoire : les questionnements de Lavoisier et Laplace]{Un peu d’histoire :\onlyamphibook{\\} Lavoisier et Laplace s'interrogent sur la nature de la chaleur}
\label{ch_histoire_lavoisier_laplace_depondt}

\begin{center}\textit{Par Philippe Depondt\\ \begin{small}Université Pierre et Marie Curie, Paris\end{small}}\end{center}

	Les débats sur la nature de la chaleur se sont poursuivis jusqu'à la fin du \textsc{xix}\ieme\ siècle avec l'acceptation progressive des théories atomiques. Une étape importante dans cette réflexion est fournie avec une concision et une éloquence irrésistibles dans le \textit{Mémoire sur la chaleur} (1780~\cite{lavoisierlaplace1780}) des physiciens français \wfd{Antoine Lavoisier}{Lavoisier} et \wfd{Pierre-Simon de Laplace}{Laplace} :

%handmade quote (sur 2 colonnes dans l’amphibook, {quote} est trop étroit
	\onlyframabook{\begin{quote}}
	\onlyamphibook{\begin{historyquote}} «~Les physiciens sont partagés sur la nature de la chaleur. Plusieurs d’entre eux la regardent comme un fluide répandu dans toute la nature, et dont les corps sont plus ou moins pénétrés, à raison de leur température et de leur disposition particulière à le retenir ; il peut se combiner avec eux, et, dans cet état, il cesse d’agir sur le thermomètre et de se communiquer d’un corps à l’autre, ce n’est que dans l’état de liberté, qui lui permet de se mettre en équilibre dans les corps, qu’il forme ce que nous nommons \textit{chaleur libre}.

	D’autres physiciens pensent que la chaleur n’est que le résultat des mouvements insensibles des molécules de la matière. On sait que les corps, même les plus denses, sont remplis d’un grand nombre de pores ou de petits vides, dont le volume peut surpasser considérablement celui de la matière qu’ils renferment ; ces espaces vides laissent à leurs parties insensibles la liberté d’osciller dans tous les sens, et il est naturel de penser que ces parties sont dans une agitation continuelle, qui, si elle augmente jusqu’à un certain point, peut les désunir et décomposer les corps ; c’est ce mouvement intestin qui, suivant les physiciens donc nous parlons, constitue la chaleur.

	Pour développer cette hypothèse, nous ferons observer que, dans tous les mouvements dans lesquels il n’y a point de changement brusque, il existe une loi générale que les géomètres ont désignée sous le nom de \textit{principe de la conservation des forces vives} ; cette loi consiste en ce que, dans un système de corps qui agissent les uns sur les autres d’une manière quelconque, la force vive, c’est-à-dire la somme des produits de chaque masse par le carré de sa vitesse, est constante. Si les corps sont animés par des forces accélératrices, la force vive est égale à ce qu’elle était à l’origine du mouvement, plus à la somme des masses multipliées par les carrés des vitesses dues à l’action des forces accélératrices. Dans l’hypothèse que nous examinons, la chaleur est la force vive qui résulte des mouvements insensibles des molécules d’un corps ; elle est la somme des produits de la masse de chaque molécule par le carré de sa vitesse.

	Si l’on met en contact deux corps dont la température soit différente, les quantités de mouvements qu’ils se communiqueront réciproquement seront d’abord inégales ; la force vive du plus froid augmentera de la même quantité dont la force vive de l’autre diminuera, et cette augmentation aura lieu jusqu’à ce que les quantités de mouvement communiquées de part et d’autre soient égales ; dans cet état la température des corps sera parvenue à l’uniformité.

	Cette manière d’envisager la chaleur explique facilement pourquoi l’impulsion directe des rayons solaires est inappréciable, tandis qu’ils produisent une grande chaleur. Leur impulsion est le produit de leur masse par leur simple vitesse ; or, quoique cette vitesse soit excessive, leur masse est si petite, que ce produit est presque nul, au lieu que leur force vive étant le produit de leur masse par le carré de leur vitesse, la chaleur qu’elle représente est d’un ordre très-supérieur à celui de leur impulsion directe. Cette impulsion sur un corps blanc, qui réfléchit abondamment la lumière, est plus grande que sur un corps noir, et cependant les rayons solaires communiquent au premier une moindre chaleur, parce que ces rayons, en se réfléchissant, emportent leur force vive, qu’ils communiquent au corps noir qui les absorbe.

	Nous ne déciderons point entre les deux hypothèses précédentes ; plusieurs phénomènes paraissent favorables à la dernière ; tel est, par exemple, celui de la chaleur que produit le frottement de deux corps solides ; mais il en est d’autres qui s’expliquent plus simplement dans la première ; peut-être ont-elles lieu toutes deux à la fois. Quoi qu’il en soit, comme on ne peut former que ces deux hypothèses sur la nature de la chaleur, on doit admettre les principes qui leur sont communs ; or, suivant l’une et l’autre, \emph{la quantité de chaleur libre reste toujours la même dans le simple mélange des corps}. Cela est évident, si la chaleur est un fluide qui tend à se mettre en équilibre, et, si elle n’est que la force vive qui résulte du mouvement intestin de la matière, le principe dont il s’agit est une suite de celui de la conservation des forces vives. La conservation de la chaleur libre, dans le simple mélange des corps, est donc indépendante de toute hypothèse sur la nature de la chaleur ; elle a été généralement admise par les physiciens, et nous l’adopterons dans les recherches suivantes.~» \onlyframabook{\end{quote}}\onlyamphibook{\end{historyquote}}

	À l'époque où ce texte a été écrit, l'hypothèse atomique restait largement spéculative, faute de moyens expérimentaux adéquats : l'expérience de Jean Perrin qui a finalement tranché n'a eu lieu que dans les premières années du \textsc{xx}\ieme\ siècle, et les expériences de diffraction de rayons X suggérées par Max von Laue se situent en 1912.

\atendofhistorysection
