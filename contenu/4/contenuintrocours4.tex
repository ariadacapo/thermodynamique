Au cours des chapitres~\deux et~\trois nous avons appris à quantifier les transferts d’énergie, mais nous ne pouvons le faire que si nous connaissons les valeurs de $u$ ou de $h$, propriétés qu’il est impossible de mesurer directement en pratique.
	
Ce \coursquatre se propose de répondre à deux questions :
\begin{itemize}
	\item Comment peut-on décrire le comportement de l’air lorsqu’il est chauffé ou comprimé ?
	\item Comment peut-on prévoir les valeurs de $u$ et de $h$ lorsqu’on utilise de l’air ?
\end{itemize}
Ce chapitre est incompatible avec le \courscinq, où nous devrons oublier tout ce qui est appris ici.
