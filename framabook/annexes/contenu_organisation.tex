\clearpage
\widecenteredpagegeometry
\pagestyle{empty}

\chapter*{Organisation de l’ouvrage} % comme tout le monde s’en fout, on ne la met pas dans la TOC
\thispagestyle{empty}

%TODO
TODO: reformuler ces objectifs en décrivant de façon plus complète la structure de l’ouvrage (la façon dont il remplit ces objectifs)

Donner à l’étudiant/e les moyens de décrire et de quantifier :
	\begin{itemize}
		\item le comportement des fluides lors des transferts de chaleur et de travail ;
		\item le principe de fonctionnement des moteurs et réfrigérateurs ;
		\item les principales caractéristiques des moteurs de l’industrie.
	\end{itemize}

Le livre est abordable avec un niveau Baccalauréat, et peut servir d’appui pour aborder ensuite un cours de mécanique des fluides ou de conception motorisation. Il n’est pas destiné à la préparation d’un concours \textit{prépa}, mais il peut servir pour consolider ou re-visiter les notions qui y sont abordées.

\restoregeometry
\pagestyle{fancy}
