\thispagestyle{empty}
\newgeometry{left=3.5cm,right=3.5cm,top=2cm,bottom=2cm}

	\vspace*{\stretch{1}}

	\adjustbox{center}{\swrule{1.025\textwidth}{1pt}}

	\vspace{-0.5cm}
	{\center \normalsize \huge Objectifs\par}
	\vspace{0.5cm}

	Donner à l’étudiant/e les moyens de décrire et de quantifier :
	\begin{itemize}
		\item le comportement des fluides lors des transferts de chaleur et de travail ;
		\item le principe de fonctionnement des moteurs et réfrigérateurs ;
		\item les principales caractéristiques des moteurs de l’industrie.
	\end{itemize}

	Le livre est abordable avec un niveau Baccalauréat, et peut servir d’appui pour aborder ensuite un cours de mécanique des fluides ou de conception motorisation. Il n’est pas destiné à la préparation d’un concours \textit{prépa}, mais il peut servir pour consolider ou re-visiter les notions qui y sont abordées.

	\vspace{0.3cm}
	\adjustbox{center}{\swrule{1.025\textwidth}{1pt}}

	\vspace*{\stretch{1}}

\restoregeometry
\pagestyle{fancy}




